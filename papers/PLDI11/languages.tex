% !TEX TS-program = pdflatex
% !TEX encoding = UTF-8 Unicode

\documentclass[preprint]{sigplanconf}

\usepackage{graphicx,listings,fixltx2e,lambda,array,multirow,color}


\begin{document}
\conferenceinfo{PLDI 2011}{June 4--8, 2011, San Jose, CA, USA.}
\copyrightyear{2010}

\preprintfooter{PLDI 2011}
\titlebanner{DRAFT---Do not distribute}

\title{Parakeet: Fast GPU Programming with Array Languages} 
% EDIT: old title was a bit wordy, looked gross, and "high level" is redundant if someone even glances at our abstract. 
%  --- General Purpose GPU Programming with High Level ArrayLanguages}

\authorinfo{Anonymous}{Anonymous}{Anonymous}
\lstset{ 
basicstyle=\small\footnotesize\ttfamily, % Standardschrift
numbers=left,                   % where to put the line-numbers
stepnumber=1,                   % the step between two line-numbers. If it's 1 each line 
numbersep=12pt,                  % how far the line-numbers are from the code
showspaces=false,               % show spaces adding particular underscores
showstringspaces=false,         % underline spaces within strings
showtabs=false,                 % show tabs within strings adding particular underscores
tabsize=2,	                % sets default tabsize to 2 spaces
captionpos=b,                   % sets the caption-position to bottom
breaklines=true,                % sets automatic line breaking
breakatwhitespace=false,        % sets if automatic breaks should only happen at whitespace
keywordstyle=\color{black}\bf,
morekeywords={til,sum,sqrt,all, min, each, where, while,+,*,-,avg},            % if you want to add more keywords to the set
moredelim=[is][\itshape]{/*}{*/},
linewidth={\textwidth},
xleftmargin=20pt,
%framexleftmargin=17pt,
%framexrightmargin=5pt,
%framexbottommargin=4pt,
}

% define some useful commands to use in language specification 
\newcommand{\WITH}{\impfnt{with}}
\newcommand{\VALUES}{\impfnt{values}}
\newcommand{\MAP}{\impfnt{map}}
\newcommand{\REDUCE}{\impfnt{reduce}}
\newcommand{\ALLPAIRS}{\impfnt{allpairs}}
\newcommand{\CAST}{\impfnt{cast}}
\newcommand{\APPLY}{\impfnt{apply}}
\newcommand{\INDEX}{\impfnt{index}}
\newcommand{\SCAN}{\impfnt{scan}}
\newcommand{\THREADIDX}{\impfnt{threadIdx}}
\newcommand{\BLOCKIDX}{\impfnt{blockIdx}}
\newcommand{\BLOCKDIM}{\impfnt{blockDim}}
\newcommand{\GRIDDIM}{\impfnt{gridDim}}
\newcommand{\VEC}{\impfnt{vec}}
\newcommand{\SET}{\impfnt{set}}
\newcommand{\CALL}{\impfnt{call}}
\newcommand{\CLOSURE}{\impfnt{closure}}
\newcommand{\PRIM}{\impfnt{prim}}
\newcommand{\concat}{\ensuremath{+\!\!\!\!+\,}}

\setlength\fboxsep{8pt}
\setlength\fboxrule{0.5pt}

\maketitle
\section{Untyped Higher-Order IL}
\begin{figure}[h!]
  \begin{tabular}{| m{0.01cm}m{1.5cm}m{0.1cm}m{0.2cm}p{4.5cm} |}
  \hline
  & & & &\\ 
   \multicolumn{5}{|l|}{\textbf{Untyped Higher-Order IL}}  \\[4pt]
  & program & $p$ &  $\bnfdef$   &  $d_1 \cdots d_n $ \\[4pt]
  & definition & $d$ & $\bnfdef$ & $f_i(\overline{x}) \rightarrow (\overline{y}) = \overline{s}$ \\[4pt]
  & statement  & $s$ & $\bnfdef$ & $\overline{x} = e $\\[2pt]
  &            &     & $\sep$    & $\IF ~v~ \THEN ~ \overline{s} ~ \ELSE ~ \overline{s} $ \\[2pt]
  &            &     & $\sep$    & $\WHILE ~\overline{s}, x_{cond} ~ \DO ~\overline{s}~ $  \\[5pt]
  & expression & $e$ & $\bnfdef$ & $\VALUES (\overline{v})$ \\[2pt]
  &            &     & $\sep$    & $ v_f (\overline{v}) $ \\[5pt]
  &            &     & $\sep$    & $\MAP(v_f, \overline{v})$ \\[2pt]
  &            &     & $\sep$    & $\REDUCE(v_f, \overline{v}_{acc}, \overline{v})$ \\[2pt]
  &            &     & $\sep$    & $\SCAN(v_f, \overline{v}_{acc}, \overline{v})$ \\[9pt]

  & value      & $v$ & $\bnfdef$ & numeric constant\\[2pt]
  &            &     & $\sep$    &  $x$  \quad \small{(data variable)} \\[2pt]
  &            &     & $\sep$    &  $f$  \quad \small{(function label)} \\[2pt]
  &            &     & $\sep$    &  $\oplus$ \quad \small{(primitive operator)} \\[5pt]
  \hline
  \end{tabular}
\caption{Untyped Higher-Order Intermediate Language}
\end{figure}


\section{Typed First-Order IL}
\begin{figure}[h!]
  \begin{tabular}{| m{0.01cm}m{1.3cm}m{0.1cm}m{0.2cm}p{5.0cm} |}
  \hline
    & & & &\\
   \multicolumn{5}{|l|}{\textbf{Typed First-Order IL}}  \\[4pt]
  & program & $p$ &  $\bnfdef$   &  $d_1 \cdots d_n $ \\[4pt]
  & definition & $d$ & $\bnfdef$ & $f_i(x^m : \tau^m) \rightarrow (y^n : \tau^n) = \overline{s} $ \\[4pt]
  & statement  & $s$ & $\bnfdef$ & $x^m : \rho^m = e $\\[2pt]
  &            &     & $\sep$    & $\IF ~v~ \THEN ~\overline{s}~ \ELSE ~ \overline{s}$ \\[2pt]
  &            &     & $\sep$    & $\textbf{fncase} ~ f_i, \overline{x} ~\small{\Rightarrow}~ \overline{s} ~ | ~\small{\cdots}~ | ~ f_j, \overline{x} ~\small{\Rightarrow}~ \overline{s}   $ \\[2pt]
  &            &     & $\sep$    & $\WHILE ~ \overline{s}, x_{cond} ~ \DO ~ \overline{s} ~  $ \\[4pt]
  & expression & $e$ & $\bnfdef$ & $\VALUES(\overline{v})$ \\[2pt]
  &            &     & $\sep$    & $\CLOSURE _{\left\langle f_i \right\rangle} (v^c) $ \\[1.5pt]
  &            &     &           & ~~\small{where arity($f_i$) $ \small{>}  c$} \\[2pt]
  &            &     & $\sep$    & $\PRIM_{\left\langle \oplus \right\rangle }(v^m)$ \\[1.5pt]
  &            &     &           & ~~\small{where arity($\oplus$) = $m$} \\[2pt]
  &            &     & $\sep$    & $\CALL _{\left\langle f_i  \right\rangle } (v^m)$ \\[1.5pt] 
  &            &     &           & ~~\small{where arity($f_i$) = $m$} \\[2pt]
  &            &     & $\sep$    & $\MAP _{\left\langle f_i, c \right\rangle}(v^c, v^m)$ \\[1.5pt]
  &            &     &           & ~~\small{where arity($f_i$) = $c+m$} \\[2pt]
  &            &     & $\sep$    & $\REDUCE _{ \left\langle f_i, c_i, f_r, c_r \right\rangle } (v^{c_i}, v^n, v^{c_r}, v^{m})$ \\[1.5pt]
  &            &     &           & ~~\small{where $f_i : \tau^{c_i + n} \rightarrow \tau^n$} \\[1.5pt]
  &            &     &           & ~~~~~~~~~~~ \small{$f_r : \tau^{c_r + m + n} \rightarrow \tau^n$} \\[2pt]
  &            &     & $\sep$    & $\SCAN _{ \left\langle f_i, c_i, f_r, c_r \right\rangle} (v^{c_i}, v^n, v^{c_r}, v^{m})$ \\[1.5pt]
  &            &     &           & ~~\small{where $f_i : \tau^{c_i+n} \rightarrow \tau^n$} \\[1.5pt]
  &            &     &           & ~~~~~~~~~~~\small{$f_r : \tau^{c_r + m + n} \rightarrow \tau^n$} \\[2pt]
  &            &     & $\sep$    & $\CAST (v, \tau_{src}, \tau_{dest})$ \\[5pt]
  & value      & $v$ & $\bnfdef$ & numeric constant\\[2pt]
  &            &     & $\sep$    &  $x$  \quad \small{(data variable)} \\[5pt]
  & & & &\\
  \multicolumn{5}{|l|}{\textbf{Type System}} \\[4pt]
  & data     & $\tau$    & $\bnfdef$ & $bool \sep int \sep float \sep \mathbf{vec} \; \tau   $ \\[2pt]
  & closures        & $\rho$  & $\bnfdef$ & $\overline{\tau} \rightarrow \overline{\tau} \sep  \tau$ \\[4pt]

  & & & &\\
  \hline
  \end{tabular}\\[4pt]
\caption{Typed First-Order Language}
\end{figure}

The only way to construct values of type $\tau^* \rightarrow \tau^*$ is through use of the $\CLOSURE$ syntax. Closures cannot, however, be called directly but rather must be deconstructed using $\textbf{fncase}$. The branches of a case statement are used to discriminate between multiple function labels and extract the components of a closure's environment. In a language with more general support for algebraic datatypes, there would be no need for both $\textbf{fncase}$ and $\textbf{if}$.  

\section{Specialization Algorithm}
\begin{lstlisting}
X
Y
Z
\end{lstlisting}

\end{document}